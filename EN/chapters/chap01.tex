
\chapter{Introduction}
\label{chapter:Introduction}

This chapter shows the format of chapter-section-subsection, et al.

\section{Background}
The discovery of seafloor hydrothermal systems in the 1970s has led to
a major reassessment of the Earth’s thermal and geochemical budgets and has revolutionized
our understanding of biological processes .   These
hydrothermal systems are found in the deep ocean along mid-ocean ridge spreading centers
where new oceanic crust is continuously created.   Hydrothermal circulation of seawater is
driven by magmatic heat which is ultimately coming from radiogenic decay within the Earth’s
deep interior. Seawater percolates into the crust, slowly heats up on its way down and reacts
with the basaltic rock.   Near the base of the system, at the magmatic-hydrothermal inter-
face, fluids are expected to reach their hottest temperature, after which they rise vigorously
and vent as hydrothermal fluids at so-called black-smokers.   When the hot fluids discharge
and contact cold seawater, minerals precipitate to form the black ”smoke” and build large
chimney structures.
Energy fluxes associated with black-smoker hydrothermal systems are large, accounting
for approximately 25\% of Earth’s total heat flux (Stein and Stein, 1994; Fisher, 2001).   To
dissipate this heat, it is estimates that the total mass of the oceans is circulated through
the crust in less than one million years (Wolery and Sleep, 1976).   These large mass fluxes
indicate   that chemical   fluxes   between   ocean and   crust,   due   to  the   reaction  of   fluid   with
the rock, are also substantial.   Therefore, black-smoker systems play a fundamental role in
regulating the chemistry of the oceans through geologic time.



\section{Direct observational constraints}
During the last decades direct observational data from both manned and unmanned sub-
mersibles has become available , giving constraints on the size of sub-seafloor convection cells and
the magnitude of energy- and mass transport. By now, active black-smoker fields have been
identified along all spreading margins, irrespective of the spreading rate, at depths ranging
from 800m to 3600m below sea level .   Generally a distinction is made
between black-smoker systems along fast spreading and slow spreading ridges.   The latter
have a substantially lower heat supply and seem controlled by tectonic rather than magmatic
processes.   In contrast to fast spreading systems, slow spreading ridges are located at much
deeper ocean depths, often lack clear magmatic melt lenses , and normally host
fewer vent fields per kilometer of ridge axis.   Nonetheless, black-smoker fields at both types
of ridges share many of the same characteristics.   Vent field size is typically on the order of
103 to 105 m2 (i.e.   one to tens of football fields) hosting few to several tens of individual
black-smoker chimneys .

\subsection{Seismic Survey}


Seismic surveys have given insight into the structure of the upper oceanic crust. A feature
common to many fast spreading ridges is an axial reflector at a depth of 1 to 2 km, generally
thought to represent the top of a melt lens .   Below this melt lens, with
a cross-ridge extent often no more than 2km, a larger mush zone has been identified, which,
together with the melt lens, supplies the magma that forms the crust . The restricted width of this mush zone suggests that fluid flow and associated
advective heat loss may occur through the full thickness of oceanic crust. Seismic studies of
slow-spreading ridges have rarely revealed shallow axial reflectors indicative of melt lenses.



