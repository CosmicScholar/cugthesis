
\chapter{绪论}
\label{chapter:Introduction}

本章主要展示章-节-小节格式。

\section{研究背景、目的和意义}
学位论文,尤其是博士学位论文少则一百多页,多则超过两百页。
如此长篇的论文如果用word编写,则会遇到很多问题。
首先是排版问题:文本字体、段落格式、行内公式、行间公式、参考文献引用、图表交叉引用等,
用word做这类事情真是一种煎熬(只是个人感受)。
但是latex可轻松搞定这一切,只需将所有注意力放在内容创作上即可。
其次是加载图片问题,首先word无法插入高分辨率的矢量图,而且文档中图片很多的时候整个操作会很卡。
而latex只需要将所有的图片放在指定的目录里面,然后在使用的地方用一句简单的命令包含进来即可,并不会影响latex文档大小。
这只是在此提到的两个主要问题,相信写过或者正在写学位论文的同学才能真正明白其中的苦衷。

本模板CugThesis latex模板是作者自己根据ThuThesis模板和中国地质大学(武汉)学位论文编写规范制定的。
以方面方便自己的博士论文编写,另一方面可以将其分享给地大的学子们,当然了是有兴趣用latex编写论文的同学!
但在此申明:作者只是分享而已,根据自己情况选择,作者不对使用此模板导致的任何问题负责任!


\section{研究现状和存在问题}


\subsection{国内外研究现状和发展趋势}
在国外很多大学都有其学位论文latex模板,而且大多数大学都建议毕业生用latex编写 学位论文 \citep{andersen2017faulting}。
而国内虽然大多数大学没有这么建议,但是也已经有不少大学的毕业生自己制作latex模板 ,并在网络上分享。比如清华大学、中国科学技术大学、南京大学等。
对于中国地质大学(武汉)的学位论文latex模板,之前有地空学院的一位学长做过latex模板,
但是本人并没有测试成功(可能是本人对齐设置有误等原因),也有计算机学院等同学做了相应的模板。


\subsection{存在问题和发展趋势}
latex编写论文,唯一的缺点就是审阅模式问题。
这也是latex写作很难推广的一个主要原因,
与word审阅模式不同 ,latex主流的免费编辑器(比如TeXstudio并不支持审阅模式)。
所以如果导师非要用 word审阅模式的方式来修改论文,那么这种方式就是个很大的问题。
否则,用latex写学位论文只有高效和稳定,尤其是再用git进行 版本控制!

参考文献使用实例: \cite{coumou2006dynamics}研究了海底热液系统的三维模拟。


